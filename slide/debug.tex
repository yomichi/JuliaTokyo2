\section{デバッグ}

\begin{frame}
  \frametitle{}
  {\Huge デバッグツール}
\end{frame}

\begin{frame}[containsverbatim]
\frametitle{デバッグ用ツール}
\begin{itemize}
  \item \verb|Base.@show|
    \begin{itemize}
      \item \verb|printf| デバッグ用マクロ
      \item 式そのものと、その値を表示できる
        \begin{itemize}
          \item 複数引数も、タプルという有効な式なので問題なく表示できる
        \end{itemize}
      \item 必要タイプ数も少ないし、ほぼデバッグ専用なので\verb|grep|も楽
      \item 神マクロのひとつ
    \end{itemize}
    \begin{lstlisting}
julia> x = 42;
julia> @show x;
x = 42
julia> @show x,2x;
(x,2x) = (42,84)
\end{lstlisting}
\end{itemize}
\end{frame}

\begin{frame}[containsverbatim]
\frametitle{デバッグ用ツール}
\begin{itemize}
  \item \verb|Base.@assert|
    \begin{itemize}
      \item 渡した式の値が偽なら式を表示して強制終了
      \item 防衛的プログラミング用
\begin{lstlisting}
julia> @assert x == 42
julia> @assert x != 42
ERROR: assertion failed: x != 42
 in error at error.jl:21
 \end{lstlisting}
 \item フラグ1つで\verb|@assert| を全部無効化する、といった機能は実はない
   \begin{itemize}
     \item こないだissue ができたし\footnote{\#7732}、いつかできるんじゃないでしょうか
     \item \verb|Base.@assert| を書き換えることでむりやり無効化できる
       \begin{lstlisting}
julia> import Base.@assert
julia> macro assert(ex)
       :nothing
       end
       \end{lstlisting}
   \end{itemize}
\end{itemize}
\end{itemize}
\end{frame}

\begin{frame}[containsverbatim]
\frametitle{デバッグ用ツール}
\begin{itemize}
 \item \verb|Debug.jl|
   \begin{itemize}
     \item 対話型デバッガ
     \item ステップ実行や変数参照などそれなりに出来る
     \item デバッグしたいブロックに予め\verb|@debug| をつけておく
     \item ブレークポイントを予め付けたい場合\verb|@bp| をつけておく
     \item \verb|gdb| 含めあんまり使っていないのでなんとも……
   \end{itemize}
 \item \verb|Logging.jl|
   \begin{itemize}
     \item ログ記録用ツール
     \item ログ情報にレベルを付与することができる
     \item 実行レベルを変えることで残す情報を取捨選択できる
   \end{itemize}
\end{itemize}
\end{frame}

