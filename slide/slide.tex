\documentclass[dvipdfmx]{beamer}
\usepackage{mytheme}
\setbeamercovered{transparent=0}

\usepackage{listings, jlisting}
\lstset{%
 language={python},
 morekeywords={end, function},
 backgroundcolor={\color[gray]{.9}},%
 basicstyle={\small\ttfamily},%
 identifierstyle={\small\ttfamily},%
 commentstyle={\small\ttfamily\color[gray]{.50}},%
 keywordstyle={\bfseries},%
 ndkeywordstyle={\small\ttfamily},%
 stringstyle={\small\ttfamily},
 frame={tb},
 breaklines=true,
 %columns=[l]{fullflexible},%
 numbers=left,%
 xrightmargin=0zw,%
 xleftmargin=3zw,%
 numberstyle={\scriptsize},%
 stepnumber=1,
 numbersep=1zw,%
 lineskip=-0.5ex%
}



\title{Julia 開発の道具箱(チューニング編)}
\author{Yuichi Motoyama}
\date{JuliaTokyo \#2 2014/09/27}

\begin{document}

\begin{frame}
  \titlepage
  スライドとサンプルコード \\ 
  https://github.com/yomichi/JuliaTokyo2
\end{frame}

\begin{frame}[containsverbatim]
  \frametitle{自己紹介}
  \begin{columns}
    \begin{column}{.8\linewidth}
  \begin{itemize}
    \item HN : 夜道, yomichi
    \item twitter : \verb|@yomichi_137|
    \item ぽすどくいちねんせい
      \begin{itemize}
        \item 統計物理学・計算物理学
        \item 主にモンテカルロ法とか局在スピン系とか
      \end{itemize}
    \item イベント(勉強会)実況勢
      \begin{itemize}
        \item 最近だと全ゲ連(同人ゲーム開発)とか、PyConJP とか、JuliaTokyoとか
        \item 自分で自分の発表は実況できないのがつらい
      \end{itemize}
    \item Julia nightly build 勢
    \item Julia の開発環境は REPL と Vim
    \item 去年の夏コミ・冬コミでJulia 本出してました
      \begin{itemize}
        \item blog とか締め切りがないので書けません><
        \item 抽選受かっていたら次の冬にも出します
      \end{itemize}
  \end{itemize}
\end{column}
\begin{column}{.3\linewidth}
  \includegraphics[width=\linewidth]{sakuremi.jpg}

  (thanks \verb|@am_11|)
\end{column}
  \end{columns}
\end{frame}

\begin{frame}
  \frametitle{今日のお話}
  Julia 言語で書いたスクリプトの改善、つまり
  \begin{itemize}
    \item デバッグ
    \item リファクタリング
    \item チューニング
  \end{itemize}
  をするのに便利なツールを紹介!!
  \onslide<2>

  という予定でしたが、嬉しい事に思ったより話すことがあって時間がなさそうな気がするので、
  前2つはツールの紹介だけ軽くして、今日は主にチューニングの話をします><
  \onslide
\end{frame}

\section{デバッグ}

\begin{frame}
  \frametitle{}
  {\Huge デバッグツール}
\end{frame}

\begin{frame}[containsverbatim]
\frametitle{デバッグ用ツール}
\begin{itemize}
  \item \verb|Base.@show|
    \begin{itemize}
      \item \verb|printf| デバッグ用マクロ
      \item 式そのものと、その値を表示できる
        \begin{itemize}
          \item 複数引数も、タプルという有効な式なので問題なく表示できる
        \end{itemize}
      \item 必要タイプ数も少ないし、ほぼデバッグ専用なので\verb|grep|も楽
      \item 神マクロのひとつ
    \end{itemize}
    \begin{lstlisting}
julia> x = 42;
julia> @show x;
x = 42
julia> @show x,2x;
(x,2x) = (42,84)
\end{lstlisting}
\end{itemize}
\end{frame}

\begin{frame}[containsverbatim]
\frametitle{デバッグ用ツール}
\begin{itemize}
  \item \verb|Base.@assert|
    \begin{itemize}
      \item 渡した式の値が偽なら式を表示して強制終了
      \item 防衛的プログラミング用
\begin{lstlisting}
julia> @assert x == 42
julia> @assert x != 42
ERROR: assertion failed: x != 42
 in error at error.jl:21
 \end{lstlisting}
 \item フラグ1つで\verb|@assert| を全部無効化する、といった機能は実はない
   \begin{itemize}
     \item こないだissue ができたし\footnote{\#7732}、いつかできるんじゃないでしょうか
     \item \verb|Base.@assert| を書き換えることでむりやり無効化できる
       \begin{lstlisting}
julia> import Base.@assert
julia> macro assert(ex)
       :nothing
       end
       \end{lstlisting}
   \end{itemize}
\end{itemize}
\end{itemize}
\end{frame}

\begin{frame}[containsverbatim]
\frametitle{デバッグ用ツール}
\begin{itemize}
 \item \verb|Debug.jl|
   \begin{itemize}
     \item 対話型デバッガ
     \item ステップ実行や変数参照などそれなりに出来る
     \item デバッグしたいブロックに予め\verb|@debug| をつけておく
     \item ブレークポイントを予め付けたい場合\verb|@bp| をつけておく
     \item \verb|gdb| 含めあんまり使っていないのでなんとも……
   \end{itemize}
 \item \verb|Logging.jl|
   \begin{itemize}
     \item ログ記録用ツール
     \item ログ情報にレベルを付与することができる
     \item 実行レベルを変えることで残す情報を取捨選択できる
   \end{itemize}
 \item \verb|SIUnits.jl|
   \begin{itemize}
     \item 型の力でバグを防ぐことを目指すパッケージのひとつ
       \begin{itemize}
         \item いわゆる次元解析
         \item 例えば質量と長さは、普通のプログラミングではともに普通の実数として表現されるが、本質的には別物
         \item 加減は許されないが乗除は許され、さらに新たな単位を作る
       \end{itemize}
     \item 速度・メモリ上のオーバーヘッドも無い
     \item 実際に使うには、SI 単位系を強制されるので正直窮屈……
   \end{itemize}
\end{itemize}
\end{frame}


\section{リファクタリング}

\begin{frame}
  \frametitle{}
  {\Huge リファクタリングツール}
\end{frame}

\begin{frame}[containsverbatim]
\frametitle{リファクタリング用ツール}
\begin{itemize}
  \item \verb|Lint.jl|
    \begin{itemize}
      \item Julia 版 \verb|lint|
      \item 未使用変数とか \verb| if a = 0 | とかそういう怪しいコードを教えてくれる
      \item 現在も更新がかなり盛ん
    \end{itemize}
    \begin{lstlisting}
julia> using Lint

shell> cat ex_Lint.jl
let x = 42
end

julia> lintfile("ex_Lint.jl")
          ex_Lint.jl [               ]    0 WARN   Local vars declared but not used: x
    \end{lstlisting}
\end{itemize}
\end{frame}

\begin{frame}[containsverbatim]
\frametitle{リファクタリング用ツール}
\begin{itemize}
  \item \verb|TypeCheck.jl|
    \begin{itemize}
      \item 型の一貫性などをチェックできる
      \item コンパイラが最適化してくれるかどうかに効いてくる
        \begin{description}
          \item[checkreturntypes] 関数の返り値の型が全部同じか
          \item[checklooptypes] ループの中で変数の型が不変か
          \item[code\_typed] 各変数が取りうる型を併記して関数の定義を表示
        \end{description}
    \end{itemize}
\end{itemize}
\end{frame}


\section{チューニング}

\begin{frame}
  \frametitle{}
  {\Huge チューニング}
\end{frame}

\begin{frame}
\frametitle{チューニング?なにそれおいしいの?}
\begin{itemize}
  \item プログラム自体の性能最適化
    \begin{itemize}
    \onslide<2-> \item なにを最適化したいのか考えてありますか?
      \begin{itemize}
        \item 大抵は計算時間の最小化か、要求メモリの最小化
        \item リファクタリングも保守性の最適化と言えなくもない
      \end{itemize}
    \onslide<3-> \item そもそも問題(要求)に対して正しい答えを返すプログラムは書けましたか?
      \begin{itemize}
        \item 間違ったコードをチューニングしても正しくなることはまず無い
      \end{itemize}
    \end{itemize}
  \onslide<4-> \item 最適化にもコスト・リソースが必要
    \begin{itemize}
      \onslide<5-> \item あなたのやりたいこと、本当に割に合いますか?
      \item わざわざ大切な時間を費やし、保守性を下げてまでやる必要はありますか?
      \onslide<6-> \item 他にやることはないのですか?
        \begin{itemize}
          \onslide<7-> \item その開発時間を使って新機能を追加する
          \onslide<8-> \item その開発時間を計算時間そのものに充てる
          \onslide<9->\item その開発時間の分、早く帰って寝る
        \end{itemize}
      \onslide<10-> \item これから何度も使う、とっておきのコードは頑張って磨く価値がある
    \end{itemize}
\end{itemize}
\end{frame}

\begin{frame}
\frametitle{格言集}
{\Large
\begin{itemize}
  \item Donald Knuth
  \begin{itemize}
    {\Large
    \onslide<2-> \item 時期尚早な最適化は諸悪の根源である \onslide
    }
  \end{itemize}
  \item Michael A. Jackson
  \begin{description}
    \onslide<3-> \item [最適化の第一法則] \onslide<4-> 最適化するな。 \onslide
    \onslide<3-> \item [最適化の第二法則] \onslide<5-> まだするな。(上級者限定) \onslide
  \end{description}
\end{itemize}
}
\onslide<6->
\begin{itemize}
  \item だからといって、明らかにダメなコードを放置していても良いわけじゃない
  \item 時間をかけてひねったコードを産み出してまで「最適化」するのは後にしろということ
    \begin{itemize}
      \onslide<7->{\tiny \item  Knuth先生もこの後すぐに「それでも、クリティカルな3\%は見逃すな」と言っているし}
    \end{itemize}
  \onslide<8-> \item {\normalsize せっかくJulia ちゃんを使うんだから、その速度を引き出してあげよう!}
\end{itemize}
\end{frame}

\begin{frame}
  \frametitle{一般的な最適化のやりかた}
    大前提として、正しく動くプログラム
      (テストや要求仕様などに適合する)を書き上げる
  \begin{enumerate}
    \onslide<2-> \item プロファイラでホットスポットを探す
      \begin{itemize}
        \item 大抵予想外のところにあるのでツールを使うこと
        \item 全体的に同じぐらい遅いときは諦めよう
      \end{itemize}
    \onslide<3-> \item チューニングを試みる
      \begin{itemize}
        \item アルゴリズムを変える
        \item ライブラリや関数を変える
        \item コンパイラやハードウェアの気持ちになって考える
          \begin{itemize}
            \item 計算機様が気持ちよく仕事できるようなコードを書く
          \end{itemize}
      \end{itemize}
    \onslide<4-> \item テストを行って、エンバグしていないことを確かめる
    \onslide<5-> \item 時間を測って効果を確認する
    \onslide<6-> \item 妥協できるところまで繰り返す
  \end{enumerate}
\end{frame}

\begin{frame}[containsverbatim]
\frametitle{Julia における性能測定ツール}
\begin{itemize}
  \item 以下\verb|Base| 直下にある標準ツール
  \item 時間・メモリ両用
    \begin{itemize}
      \item \verb|@time| / \verb|@timed|
    \end{itemize}
  \item 時間
    \begin{itemize}
      \item \verb|@elapsed|
      \item \verb|tic| / \verb|toc| / \verb|toq|
      \item \verb|time_ns|
    \end{itemize}
  \item メモリ
    \begin{itemize}
      \item \verb|@allocated|
    \end{itemize}
  \item プロファイラ
    \begin{itemize}
      \item \verb|Profile| モジュール
      \item \verb|--track-allocation=user|
    \end{itemize}
\end{itemize}
\end{frame}

\begin{frame}[containsverbatim]
  \frametitle{時間・メモリ測定ツール}
  \begin{itemize}
    \item \verb|@time(ex)| / \verb|@timed(ex)|
      \begin{itemize}
        \item 与えた式\verb|ex| に関して、実行時間・要求メモリ・ガベコレ時間を測定する
        \item \verb|@time| は測定結果を標準出力に出し、\verb|@timed| は返り値として返す
        \item 基本的にはこれらを使えばOK
      \end{itemize}
  \end{itemize}
  \begin{lstlisting}
julia> @time sum(ones(1<<24))
elapsed time: 0.094370607 seconds (134217888 bytes allocated, 24.31% gc time)  # STDOUT
1.6777216e7  # 返り値

julia> @timed sum(ones(1<<24))
(1.6777216e7,0.101337356,134217888,0.022860107)  # 返り値
  \end{lstlisting}
\end{frame}

\begin{frame}[containsverbatim]
  \frametitle{時間測定ツール}
  \begin{itemize}
    \item \verb|@elapsed(ex)|
      \begin{itemize}
        \item 式\verb|ex| を実行し、かかった時間を秒単位で返す
        \item \verb|ex| の値そのものは返さない(仕様)
      \end{itemize}
  \end{itemize}
  \begin{lstlisting}
julia> @elapsed sum(ones(1<<24))
0.09623339   # 返り値
  \end{lstlisting}
\end{frame}

\begin{frame}[containsverbatim]
  \frametitle{時間測定ツール}
  \begin{itemize}
    \item \verb|tic()|
      \begin{itemize}
        \item 時間測定の基点をセットする
      \end{itemize}
    \item \verb|toq()|
      \begin{itemize}
        \item \verb|tic()| からの経過時間を秒単位で返す
        \item \verb|toq()| すると\verb|tic()| で作った基点はなくなる
      \end{itemize}
    \item \verb|toc()|
      \begin{itemize}
        \item \verb|toq()| して、経過時間を\verb|println| する
      \end{itemize}
  \end{itemize}
  \begin{lstlisting}
julia> tic()
0x0000a45b7e3a65c3  # 後述するtime_ns() の返り値

julia> toc()
elapsed time: 3.471314971 seconds  # STDOUT
3.471314971  # 返り値

julia> toc()
ERROR: toc() without tic()
 in toq at util.jl:28
 in toc at util.jl:36
  \end{lstlisting}
\end{frame}

\begin{frame}[containsverbatim]
  \frametitle{時間測定ツール}
  \begin{itemize}
    \item \verb|time_ns|
      \begin{itemize}
        \item 現在時刻をナノ秒単位で返す
        \item 返り値の型は \verb|UInt64|
        \item ここまで紹介した関数・マクロは内部でこれを呼ぶ
        \item ゼロ点が具体的にどこかは定められていない
          \begin{itemize}
            \item 5.8 年周期でオーバーフローするため
          \end{itemize}
      \end{itemize}
  \end{itemize}
  \begin{lstlisting}
julia> time_ns()
0x0000a44ebf7ad040

julia> (time_ns() - ans)/1e9
1.728456695
\end{lstlisting}
\end{frame}

\begin{frame}[containsverbatim]
\frametitle{メモリ測定ツール}
\begin{itemize}
  \item \verb|@allocated(ex)|
    \begin{itemize}
      \item 式\verb|ex| を実行し、その間に確保したメモリをByte 単位で返す
      \item \verb|ex| の値そのものは返さない(仕様)
      \item \verb|@timed(ex)| のメモリ版
    \end{itemize}
\end{itemize}
\begin{lstlisting}
julia> @allocated [1:1024]
8256

julia> @allocated []
48
\end{lstlisting}
\begin{itemize}
  \item 空配列を作るだけでもオーバーヘッドがあるっぽい
\end{itemize}
\end{frame}

\begin{frame}[containsverbatim]
\frametitle{プロファイラ}
\begin{itemize}
  \item \verb|Base.Profile| モジュール
    \begin{itemize}
      \item 統計的プロファイラ
      \item 周期的に(標準では1ミリ秒) どの行にいるのかを調べることで、どこでどのくらいの時間を費やしたのかを調べる
      \item バックトレースを見ているので、どの関数呼び出しから呼ばれているのかも区別可能
    \end{itemize}
\end{itemize}
\end{frame}

\begin{frame}[containsverbatim]
\frametitle{プロファイラ}
\begin{lstlisting}
julia> function prof(N::Int=100)
       xs = rand(N,N,N)
       x = maximum(xs)
       ys = rand(N,N,N)
       y = maximum(ys)
       x - y
       nothing
       end
\end{lstlisting}
\begin{itemize}
  \item この関数を調べてみる(公式ドキュメントの例をほんの少しだけ複雑にした)
\end{itemize}
\end{frame}

\begin{frame}[containsverbatim]
\frametitle{プロファイラ}
\begin{lstlisting}
julia> prof()  # 関数のJITコンパイル

julia> Profile.clear()  # 以前に測定していたら消去しておく

julia> @profile prof(200) # 測定
\end{lstlisting}
\end{frame}

\begin{frame}[containsverbatim]
\frametitle{プロファイラ}
\begin{lstlisting}
julia> Profile.print()
95 task.jl; anonymous; line: 96
 94 REPL.jl; eval_user_input; line: 54
  94 profile.jl; anonymous; line: 14
   43 none; prof; line: 2
    43 random.jl; rand!; line: 132
   5  none; prof; line: 3
    5 reduce.jl; _mapreduce; line: 168
     2 reduce.jl; mapreduce_impl; line: 281
     3 reduce.jl; mapreduce_impl; line: 285
   41 none; prof; line: 4
    41 random.jl; rand!; line: 132
   5  none; prof; line: 5
        ## ... 以下略
 \end{lstlisting}
 \begin{itemize}
   \item 各行は頭から観測回数、ファイル名、関数名、行数
 \end{itemize}
 \end{frame}
 
 \begin{frame}[containsverbatim]
  \frametitle{プロファイラ}
  \begin{lstlisting}
julia> Profile.print(format = :flat)
 Count File       Function         Line
    94 REPL.jl    eval_user_input    54
     1 REPL.jl    eval_user_input    56
    43 none       prof                2
     5 none       prof                3
    41 none       prof                4
     5 none       prof                5
    94 profile.jl anonymous          14
    84 random.jl  rand!             132
    10 reduce.jl  _mapreduce        168
     1 reduce.jl  mapreduce_impl    274
     5 reduce.jl  mapreduce_impl    281
     4 reduce.jl  mapreduce_impl    285
    95 task.jl    anonymous          96
  \end{lstlisting}
  \begin{itemize}
    \item \verb|format = :flat| とすることで、関数呼び出しツリーを無視して結果をまとめられる
  \end{itemize}
\end{frame}

\begin{frame}[containsverbatim]
\frametitle{プロファイラ}
\begin{lstlisting}
julia> open("prof.dat", "w") do io
       Profile.print(io, col=500)
       end
\end{lstlisting}
\begin{itemize}
  \item \verb|println| などのように、第一引数に\verb|IO| 型を取ることでファイル書き出しができる
    \begin{itemize}
      \item ファイルに書き出してから\verb|sort -n| や\verb|grep| をするのが便利
    \end{itemize}
  \item \verb|col| オプション引数は、出力ファイルの最大文字幅数を指定する
    \begin{itemize}
      \item 出力文字数が超過すると、ファイル名などが切り詰められるのでできるだけ大きい値を指定しておくと良い
    \end{itemize}
\end{itemize}
\end{frame}

\begin{frame}[containsverbatim]
\frametitle{プロファイラ}
\begin{itemize}
\item \verb|ProfileView.jl|
  \begin{itemize}
    \item \verb|Base.Profile| の可視化ツール
    \item 横軸がカウント、縦軸が関数呼び出しの深さ
    \item マウスオーバーでファイル名・関数名などが表示される
  \end{itemize}
\item \verb|IProfile.jl|
  \begin{itemize}
    \item instrumenting profiler
      \begin{itemize}
        \item 日本語だとなんて言うんだろう?
        \item \verb|gprof| みたいにスクリプトや実行ファイルにプロファイル用コードを埋め込んで、より精密に調べるプロファイラ
      \end{itemize}
    \item 今回ちょっと調査時間がなかった上に、
      自分のソースに使ったら落ちたのでどんな感じかよくわかりません……
  \end{itemize}
\end{itemize}
\end{frame}

 \begin{frame}[containsverbatim]
 \frametitle{メモリチェック}
 \begin{itemize}
   \item \verb|--track-allocation=user| オプション
     \begin{itemize}
       \item コードのどこでどのくらいメモリを確保しているのかを調べるオプション
       \item これをつけて\verb|Julia| を起動すると、\verb|include| した各ファイルに対してそれぞれ\verb|*.mem| ファイルが出来る
       \item それぞれの行数でどれだけ確保したのかが左に表示される
       \item 目安程度にしておいたほうがいいかも?
     \end{itemize}
 \end{itemize}
 \begin{lstlisting}
        - function alloc(N::Int)
 65425632   xs = ones(N,N,N)
        0   x = maximum(xs)
 64000064   ys = ones(N,N,N)
        0   y = maximum(ys)
        - end
        - @time alloc(200)
 \end{lstlisting}
 \begin{itemize}
   \item 関数の頭では、コンパイルや自動変数のスタックも表示されている?
 \end{itemize}
 \end{frame}

 \begin{frame}[containsverbatim]
 \frametitle{チューニングに役立つかもしれないツールとか}
 \begin{itemize}
   \item 公式ドキュメントの ``Performance Tips''
     \begin{itemize}
       \item JuliaTokyo\#1 での関さんの発表資料では実際に速度の測定をしています
       \item Tipsや紹介ツールが時々こっそりと加筆される
         \begin{itemize}
           \item \verb|--track-allocation| とかおととい初めて知った
         \end{itemize}
     \end{itemize}
   \item \verb|@inbounds|
     \begin{itemize}
       \item 配列の境界チェックを切るマクロ
       \item 普段は境界チェックに失敗すると\verb|BoundsError| 例外が飛ぶ
       \item \verb@--check-bounds={yes|no}@ というオプションで、マクロを無視してチェックを強制することも、逆に完全にチェックを外すことも可能
     \end{itemize}
 \end{itemize}
 \end{frame}

 \begin{frame}[containsverbatim]
 \frametitle{チューニングに役立つかもしれないツールとか}
 \begin{itemize}
   \item \verb|@simd|
     \begin{itemize}
       \item ループのSIMD / ベクトル並列化
       \item 最内ループのみ使える
       \item \verb|@inbounds| と組み合わせる必要がある
       \item 他にも色々な条件(仮定)が必要だが、うまくハマれば早くなる
       \item 単純なループの場合はそもそもLLVM が自動でやってくれる
     \end{itemize}
   \item \verb|@parallel for| / \verb|pmap|
     \begin{itemize}
       \item \verb|for|および\verb|map|のプロセス並列化
       \item Julia の並列処理の中では一番単純で使いやすい
       \item sfchaos さんの発表で解説があるはず……
     \end{itemize}
   \item \verb|@inline|
     \begin{itemize}
       \item v0.4-dev で登場
       \item \verb|@inline| した関数を呼び出す関数をコンパイルした時に、内側の関数をインライン展開する
       \item モジュールの最外スコープや初回呼び出しなど、コンパイルできていない場所から呼ぶ場合、逆にオーバーヘッドが発生する
     \end{itemize}
 \end{itemize}
 \end{frame}

 \begin{frame}[containsverbatim]
 \frametitle{チューニングに役立つかもしれないツールとか}
 \begin{itemize}
   \item \verb|NumericExtensions.jl|
     \begin{itemize}
       \item \verb|Base.map| やその一派を拡張・高速化するパッケージ
       \item \verb|NumericFuns.jl| 含め、数値計算する人には神パッケージ
       \item チューニング実例でも使います(時間があればやる)
     \end{itemize}
   \item \verb|NumericFuns.jl|
     \begin{itemize}
       \item 数値計算で使うけれど標準に入っていない関数など
       \item \verb|C++| でいうところの関数オブジェクトのようなものの定義
         \begin{itemize}
           \item たとえば多変数関数を無名関数を使わずに一変数関数(のようなもの)にできる
           \item 関数ではないので、関数適用は\verb|NumericFuns.evaluate(f,args...)| という別の関数になっている 
           \item 無名関数と違い、\verb|NumericFuns.evaluate| に関して
             JIT コンパイルが働くようになるので、
             特に\verb|NumericExtensions.map| などの高階関数と組み合わせるとかなり高速化される
           \item \verb|@inline| との組み合わせなども良さそう
         \end{itemize}
       \item 元々は\verb|NumericExtensions.jl| のサブモジュールだったが独立した
     \end{itemize}
 \end{itemize}
 \end{frame}

 \begin{frame}[containsverbatim]
 \frametitle{チューニング実例}
 \begin{itemize}
   \item 1次元ランダムウォークのモンテカルロ計算
   \item https://github.com/yomichi/JuliaTokyo2/src/randomwalk
   \item \verb|*_version| ディレクトリには、それぞれチューニングの各段階が入っている
     \begin{itemize}
       \item 今回はただひとつの\verb|map| を\verb|NumericExtensions.jl| で書きなおした
     \end{itemize}
   \item 各ディレクトリにある\verb|prof.jl| を実行すると時間測定とプロファイリングを行う
\begin{lstlisting}
## before (1st_version)
elapsed time: 5.182680705 seconds (803357776 bytes allocated, 5.34% gc time)

## after (4th_version)
elapsed time: 1.824523981 seconds (414096 bytes allocated)
\end{lstlisting}
\item 結論:\verb|NumericExtensions.jl| は神パッケージ
 \end{itemize}
 \end{frame}


\end{document}
